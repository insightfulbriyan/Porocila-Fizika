\documentclass[12pt]{article}


\usepackage[
    a4paper, 
    margin=2.5cm]{geometry}
\usepackage[utf8]{inputenc}         % UTF8 enkodiranje
\usepackage[slovene]{babel}          % Slovenščina
\usepackage[
    pdfusetitle, 
    hidelinks, 
    unicode]{hyperref}              % Nastavi atribute PDF-ja, ne označuj povezav
\usepackage{microtype}              % Izboljšave za tipografijsko perfekcijo :)
\usepackage{enumitem}               % Seznami za člene
\usepackage{graphicx}               % Vključitev slik
\usepackage{dirtytalk}              % Citat
\usepackage{listings}               % Kodni blok
\usepackage{fancyvrb}
\usepackage[font=]{caption}         % Required for specifying captions
\usepackage[normalem]{ulem}         % Krašanje enot v enačbi
\usepackage{times}                  % Times New Roman pisava
\usepackage{tikz} 
\usepackage[european]{circuitikz}   % Električna vezja
\usepackage{datetime}               % Datum
% \usepackage[slovenian]{csquotes}
\usepackage{braket}
\usepackage{amsmath} % matematika ki izgleda lepo
\usepackage{amsfonts} % množice
\usepackage[style=ieee, maxbibnames=3, minbibnames=1, 
    maxcitenames=1, mincitenames=1, sorting=nyt]{biblatex}   % Navajanje virov
\bibliography{viri}

\urlstyle{rm}

\setlength{\parindent}{0em}
\setlength{\parskip}{1ex}

\setcounter{secnumdepth}{5}
\setcounter{tocdepth}{4}

\renewcommand{\thesection}{\arabic{section}}
\renewcommand{\thesubsection}{\thesection.\arabic{subsection}}
\renewcommand{\thesubsubsection}{\thesubsection.\arabic{subsubsection}}
\renewcommand{\theparagraph}{\thesubsubsection.\arabic{paragraph}}
\renewcommand{\thesubparagraph}{\theparagraph.\arabic{subparagraph}}

\renewcommand{\labelnamepunct}{\addcomma\space}
\DeclareFieldFormat[article]{title}{#1}
\DeclareFieldFormat[online]{title}{\mkbibemph{#1}}

\DefineBibliographyStrings{slovene}{
  andothers = {et. al\adddot},
  urlseen = {dostopano:}
}

\newdateformat{MMYYYYdate}{\monthname[\THEMONTH] \THEYEAR}

\title{Naslov naloge}
\author{Jaka Kovač, G 4. b}

\begin{document}
\pagenumbering{arabic}

\begin{center}
    \thispagestyle{empty}
    \includegraphics[scale=1]{slike/logotip_vegova_leze_brezokvirja.png}
    \\
    \textbf{Vegova ulica 4, 1000 Ljubljana}

    \vspace{\fill} 
    Seminarska naloga pri predmetu predmet

    \Huge{\textbf{Naslov naloge}}

    \normalsize
    \vspace{\fill}

    Mentor: Mentor Priimek, prof. \hfill Avtor: Jaka Kovač, G 4. b\\
    \null
    Ljubljana, mesec leto – \MMYYYYdate\today %zamenjaj <mesec in leto> z mesecem in letom začetka
\end{center}
\newpage
\thispagestyle{empty}
\null
\newpage

\section*{Povzetek}
V tej seminarski nalogi bom predstavil kako uporabiti \LaTeX{} za pisanje seminarske naloge.
\\ %prazna vrstica
\textbf{Ključe besede:} \LaTeX{}, seminarska naloga, \LaTeX{} predloga

\vfill
\section*{Abstract}
\foreignlanguage{english}{This paper describes how to use \LaTeX{} to write a paper.
\\ %prazna vrstica
\textbf{Keywords:} \LaTeX{}, paper, \LaTeX{} template}
\vfill

% KAZALO 
\newpage
\thispagestyle{empty} % ne številčimo strani
\tableofcontents % kazalo

\begingroup     % kazalo slik
\makeatletter
\section*{Slike}
\@starttoc{lof}
\let\clearpage\relax
\makeatother
\endgroup


\newpage
\section{Uvod}
Orodje \LaTeX mi je zelo všeč zato sem se odločil, da bom napisal seminarsko nalogo v njem.
Primer seminarske naloge napisane v \LaTeX-u najdeš na https://github.com/insightfulbriyan/Komunkacijska-vezja-in-naprave.

\section{Namestitev \LaTeX-a}
    \subsection{Linux}
    Ustvarimo si mapo v katero bomo namestili \LaTeX{} in se v njo premaknemo. Najbolje je da to storiš v /home/username/Downloads.
    \begin{lstlisting}[language=bash]
        mkdir latex

        cd latex
    \end{lstlisting}

    Nato si prenesemo namestitveni paket iz https://www.tug.org/texlive/acquire-netinstall.html in ga razpakiramo.
    \begin{lstlisting}[language=bash]
        wget https://mirror.ctan.org/systems/
        texlive/tlnet/install-tl-unx.tar.gz

        tar -xzf install-tl-unx.tar.gz

        cd install-tl-*
    \end{lstlisting}

    Kot sudo zašemeno namestitveno skripto. Bodi pripravljen, popolna namestitev lahko traja nekaj ur. 
    \begin{lstlisting}[language=bash]
        sudo ./install-tl
    \end{lstlisting}
    POZOR:
    Predlagam, da namestiš vse pakete. Če veš, da nekaterih ne boš potreboval potem lahko prihraniš nekaj časa (zanemarljiv malo, glede na izkušnje)

    \subsection{Windows}
    IDK, vprašaj koga drugega. Prosim dopolni to poglavje.

    \subsection{MacOS}
    IDK, vprašaj koga drugega. Prosim dopolni to poglavje.

\newpage
\section{Uporaba \LaTeX-a}
    Spletni vodič: https://www.overleaf.com/learn/latex/Learn\_LaTeX\_in\_30\_minutes
    \begin{verbatim}
        \section{Naslov poglavja}
        \subsection{Naslov podpoglavja}
        \subsubsection{Naslov podpodpoglavja}
        \paragraph{Naslov odstavka (podpodpodpoglavja)}
        \subparagraph{Naslov pododstavka}
    \end{verbatim}

    Uporbni vodič za pisanje tehničnih besedil: http://lie.fe.uni-lj.si/Napotki\_TehnicnaBesedila.pdf \\
    dodatno pozornost posveti prvi in drugi strani ter obkroženim delom. \\
    p.s.: Vodič vsebuje tudi pogoste slovnične napake (od vključno napotka 36 dalje)
    \subsection{Slike}
    \begin{verbatim}
        \begin{figure}[h]
            \centering
            \includegraphics[scale=0.5]{slike/vegova\_logo.png}
            \caption{Vegova logo}
            \label{fig:vegova\_logo}
        \end{figure}
    \end{verbatim}

    \subsection{Formule}
    https://www.cmor-faculty.rice.edu/~heinken/latex/symbols.pdf
    \begin{equation}
        E = mc^2
    \end{equation}
    $$\frac{1}{2}$$
    $$\sqrt{2}$$
    \begin{equation}
        \frac{a}{b} = \frac{c}{d} \Rightarrow a*d = b*d
    \end{equation}

    Enačbe in formule naj bodo obdane z equation wrapperjem, izrazi načeloma ne potrebujejo.
    Posvetuj se s povesorjem, ki bo ocenjeval seminarsko nalogo. Izrazi izven wrapperja morajo biti obdani z \$\$.


\newpage
\section{BibTeX}
    \subsection{Izvoz iz COBISS-a}
    COBISS podpira izvažanje citiranja v format BibTeX. To opcijo izkoristi.
    Tudi Google Scholar podpira izvoz v BibTeX formatu.
    \subsection{Izvoz iz spletnega pomočnika}
    Predlagam: https://boberle.com/projects/bibtex-entry-generator/online/
    Lahko uporabiš tudi JabRef (https://www.jabref.org/), ki je podprt na vseh namiznih opeacijskih sistemih.
    \subsection{Samostojo pisanje}
    Spletni vodič za začetek: https://www.overleaf.com/learn/latex/Bibliography\_management\_with\_bibtex
    Drugače pa uporabi Google.




\newpage
\begingroup
\makeatletter
    \section{Viri in literatura}
    \nocite{*}
    \printbibliography[heading=none]
\makeatother
\endgroup
\newpage

\begin{samepage}
    \thispagestyle{empty}
    \section*{Izjava o avtorstvu}
    Izjavljam, da je seminarska naloga v celoti moje avtorsko delo, ki sem ga 
    izdelal samostojno s pomočjo navedene literature in pod vodstvom mentorja.

    \vfill
    
    \today \hfill Jaka Kovač
    
    \vspace{3 cm}
\end{samepage}

\end{document}
